\documentclass[12pt, a4paper]{article}

\usepackage{amsmath}
\usepackage{bm}
\usepackage{array}
\usepackage{amsmath}
\usepackage[portuguese]{babel}
\usepackage{chngpage}
\usepackage{float}
\usepackage[a4paper, margin=2cm]{geometry}
\usepackage{graphicx}
\usepackage{hyperref}
\usepackage{listings}
\usepackage{setspace}
\usepackage{xcolor}

\lstdefinestyle{codestyle}{
    commentstyle=\color{teal},
    keywordstyle=\color{blue},
    numberstyle=\ttfamily\color{gray},
    stringstyle=\color{red},
    basicstyle=\ttfamily\footnotesize,
    breakatwhitespace=false,
    breaklines=false,
    keepspaces=true,
    numbers=none,
    showspaces=false,
    showstringspaces=false,
    showtabs=false,
    tabsize=4
}
\lstset{style=codestyle}

\title{\Huge \textbf{Computação Gráfica \\ \Large Trabalho Prático -- Fase II}}
\date{30 de março 2025}
\author{Grupo 3}

\begin{document}

\begin{center}
    \includegraphics[width=0.25\textwidth]{res/cover/EE-C.eps}
\end{center}

\chardef\_=`_
\onehalfspacing
\setlength{\parskip}{\baselineskip}
\setlength{\parindent}{0pt}
\def\arraystretch{1.5}

{\let\newpage\relax\maketitle}
\maketitle
\thispagestyle{empty}

\vspace*{\fill}

\begin{adjustwidth}{-2cm}{-2cm} % These values only need to be large enough to center the table
    \begin{center}
        \begin{tabular}{>{\centering}p{0.25\textwidth}
                        >{\centering}p{0.25\textwidth}
                        >{\centering}p{0.25\textwidth}
                        >{\centering\arraybackslash}p{0.25\textwidth}}
            \includegraphics[width=3.5cm]{res/cover/A104437.png} &
            \includegraphics[width=3.5cm]{res/cover/A104348.png} &
            \includegraphics[width=3.5cm]{res/cover/A90817.png} &
            \includegraphics[width=3.5cm]{res/cover/A104179.png} \\

            Ana Oliveira & Humberto Gomes & Mariana Cristino & Sara Lopes \\
            A104437      & A104348        & A90817           & A104179
        \end{tabular}
    \end{center}
\end{adjustwidth}

\pagebreak

\begin{abstract}
    \textbf{\color{red} TODO - resumo}
\end{abstract}

\section{Transformações}

\textbf{\color{red} TODO - transformações}

\section{Modelo estático do sistema solar}

\textbf{\color{red} TODO - sistema solar}

\section{Extras}

\section{\emph{Frustum culling}}

Cada \emph{draw call} tem um custo elevado para o desempenho da aplicação. Logo, procurando reduzir
o número de \emph{draw calls}, foi implementado \emph{frustum culling}, para apenas ser requisitada
à GPU a renderização das entitidades totalmente ou parcialmente no \emph{view frustum} da câmara.

Seria muito computacionalmente intensivo verificar se a geometria de um modelo se encontra dentro ou
fora do \emph{view frustum}, pelo que se encapsulam todos os modelos em esferas que, devido à sua
geometria simples, permitem uma verificação rápida da sua visibilidade no \emph{view frustum}. No
entanto, visto que estas esferas podem ser um pouco maiores do que os modelos em si, é possível que
algumas entidades fora do ecrã sejam desenhadas, visto que parte das suas esferas ainda podem
intersetar os planos do \emph{view frustum}. Para mitigar este problema, formas geométricas
encapsuladoras mais complexas poderiam ser utilizadas, visto que estas se poderiam adaptar melhor à
geometria dos modelos. No entanto, o uso destas formas complexas conduziria a testes de visibilidade
mais caros, possivelmente anulando os benefícios de desenhar um menor número de entidades.

Quando um modelo é carregado, é necessário calcular a esfera que o encapsula. Em primeiro lugar, o
seu centro é calculado como o centro de massa de todos os pontos, como mostra a expressão abaixo,
onde $M$ é o modelo, uma sequência de pontos tridimensionais:

$$
C = \frac{1}{|M|} \sum_{p \in M} p
$$

Depois, o raio da esfera pode ser determinado como a distância entre o centro da esfera e o ponto
mais longínquo do mesmo, como mostra a expressão abaixo, onde $d$ é a função de distância cartesiana
entre dois pontos:

$$
r = \max \left \lbrace d(C, p) \mid p \in M \right \rbrace
$$

É também necessário saber como uma esfera encapsuladora é afetada quando o objeto que encapsula
sofre uma transformação no mundo. Considere-se uma matriz de transformação aplicada ao objeto (em
coordenadas do mundo), originada através da aplicação de translações, rotações, e escalas. A matriz
será 4x4 e terá o seguinte aspeto:

$$
\bgroup
    T =
    \begin{bmatrix}
        \vec \imath & \vec \jmath & \vec k & \vec t \\
        0 & 0 & 0 & 1
    \end{bmatrix}
\egroup
$$

Para calcular o centro da esfera após a transformação da entidade, basta multiplicar a matriz de
transformação pela posição do centro da esfera:

$$
C' = T C
$$

Depois, para calcular o novo raio da esfera, não é necessário ter em conta as transformações de
rotação, visto que as esferas são simétricas em todos os eixos possíveis. No entanto, é necessário
ter em conta a escala aplicada ao modelo. Por exemplo, na matriz $T$, a escala da entidade pelo
eixo $x$ é $\lVert \vec \imath \rVert$, e o mesmo se tem para o eixo $y$ e
$\lVert \vec \jmath \rVert$, e para $z$ e $\lVert \vec k \rVert$. Logo, o raio da esfera
transformada é:

$$
r' =
    \max
        \left ( \lVert \vec \imath \rVert, \lVert \vec \jmath \rVert, \lVert \vec k \rVert \right )
    \cdot
        r
$$

É possível tirar proveito da estrutura hierárquica da cena para otimizar o processo de
\emph{frustum culling}. Por exemplo, caso um grupo contenha várias entidades ou subgrupos, pode
construir-se uma esfera que encapsula a totalidade do grupo. Caso essa esfera não esteja no
\emph{view frustum}, pode-se evitar fazer os testes de visibilidade para as esferas dos objetos
individuais que compõem o grupo. Caso contrário, é na mesma necessário realizar esses testes.

O processo para determinar as características de uma esfera que encapsula todos os objetos de um
grupo é semelhante ao da construção de esferas com base no conjunto de pontos de um modelo. Em
primeiro lugar, para determinar o centro da esfera, calcula-se o centro de massa do conjunto de
pontos formado pelos centros de todas as esferas, $C$. De seguida, para cada subesfera, calcula-se
a sua distância máxima a $C$, o raio da subesfera adicionado à distância entre $C$ e o centro da
subesfera. Depois, a maior destas distâncias é escolhida para ser o raio da nova esfera, como mostra
a expressão abaixo, onde $S$ representa o conjunto de subesferas:

$$
r' = \max \left \lbrace d(C', C) + r \mid (C, r) \in S \right \rbrace
$$

Depois de saber como determinar as esferas encapsuladoras das entidades, é necessário determinar
o \emph{view fustum} da câmara, para se poder verificar a posição das esferas em relação aos planos
do \emph{view fustum}. Para determinar o \emph{view frustum}, é necessário obter alguns vetores
importantes da câmara \cite{lighthouse3d-frustum-planes}:

\textbf{\color{red} TODO - com a fusão das partes do relatório, referenciar vetores da câmara livre}

Além destes vetores, é também necessário conhecer as dimensões dos planos \emph{near} e \emph{far}.
Apesar de, matematicamente, planos não terem uma altura e uma largura, utiliza-se esta linguagem
para se referir às dimensões dos retângulos nestes planos que constituem o \emph{view frustum}. Na
expressão abaixo, mostra-se como se pode calcular a altura ($H_\text{near}$) e a largura
($W_\text{near}$) do plano \emph{near}. O processo para o plano \emph{far} é semelhante. Com o FOV
da câmara ($\theta$), a distância ao plano \emph{near} ($d_\text{near}$) e o \emph{aspect ratio}
(A), podem calcular-se as dimensões deste plano \cite{lighthouse3d-frustum-distances}:

$$
H_\text{near} = 2 d_\text{near} \tan \left ( \frac{\theta}{2} \right )
\hspace{1cm}
W_\text{near} = H_\text{near} A
$$

Com estes valores, é possível determinar as coordenadas dos pontos dos retângulos do
\emph{view frustum}. Os quatro pontos do plano \emph{near} (a amarelo na figura abaixo) podem ser
calculados do seguinte modo, e o método utilizado para o plano \emph{far} é semelhante
\cite{lighthouse3d-frustum-planes}:

$$
F = P +
    d_\text{near} \; \widehat{d} \; \pm
    \frac{H_\text{near}}{2} \; \widehat{up} \; \pm
    \frac{W_\text{near}}{2} \; \widehat{r} \;
$$

\begin{figure}[H]
    \centering
    \includegraphics[width=0.4\textwidth]{res/phase2/ViewFrustum.pdf}
    \caption{\emph{View Frustum}.}
\end{figure}

Com os oito pontos do \emph{view frustum}, é possível determinar as equações cartesianas de cada
um dos seus planos, considerando três pontos da sua superfície. Em primeiro lugar, determina-se o
vetor normal ao plano. Com os três pontos, $P_1$, $P_2$ e $P_3$, determinam-se dois vetores, a
partir dos quais se calcula um produto externo, determinando-se assim um vetor perpendicular ao
plano \cite{lighthouse3d-plane}:

$$
n = \overrightarrow{P_1 P_2} \times \overrightarrow{P_1 P_3}
$$

Depois, com este vetor normalizado, é possível determinar a constante na equação cartesiana do plano
com base nas coordenadas de um dos três pontos dados \cite{lighthouse3d-plane}:

\begin{align*}
                       & \alpha x + \beta y + \gamma z + \delta = 0 \\
    \Leftrightarrow \; & \widehat{n} \cdot P + \delta = 0 \\
    \Leftrightarrow \; & \delta = -\widehat{n} \cdot P
\end{align*}

Durante a renderização de cada \emph{frame}, verificam-se que esferas estão totalmente ou
parcialmente dentro do \emph{view frustum}. Para uma esfera estar no \emph{view frustum} $F$, é
necessário que a distância assinada entre o centro da esfera e cada um dos planos do
\emph{view frustum} não seja inferior ao simétrico do seu raio, ou seja \cite{lighthouse3d-sphere}:

$$
\forall_{p \in F} \left ( \widehat{n} \cdot C + \delta \ge - r \right )
$$

Uma vez que é necessário calcular distâncias assinadas, é imperativo que os vetores normais de todos
os planos do \emph{view frustum} apontem para o seu interior, para que o conteúdo no seu interior (e
não no seu exterior) seja desenhado. \cite{lighthouse3d-frustum-planes} Por este motivo, a ordem em
que os pontos $P_1$, $P_2$ e $P_3$ são especificados é relevante.

\section{Resultados obtidos}

Nesta fase, implementámos câmaras orbital e livre, transformações hierárquicas, frustum culling e
novas primitivas geométricas. Os resultados incluem novos modelos de primitivas gerados pelo
\texttt{generator}, testes de renderização, uma representação funcional do sistema solar e a
implementação de \textit{frustum culling}.

\subsection{Figuras geométricas geradas}

Os seguintes modelos foram criados utilizando o \texttt{generator} implementado:

\begin{figure}[H]
    \centering
    \begin{minipage}{0.48\textwidth}
        \centering
        \includegraphics[width=0.5\textwidth]{res/phase2/results/KleinBottle.png}
        \caption{Garrafa de Klein (\texttt{kleinBottle})}
    \end{minipage}\hfill
    \begin{minipage}{0.48\textwidth}
        \centering
        \includegraphics[width=0.5\textwidth]{res/phase2/results/MobiusStrip.png}
        \caption{Fita de Möbius (\texttt{möbius strip})}
    \end{minipage}
\end{figure}

\begin{figure}[H]
    \centering
    \begin{minipage}{0.48\textwidth}
        \centering
        \includegraphics[width=0.5\textwidth]{res/phase2/results/Gear.png}
        \caption{Roda dentada (\texttt{gear})}
    \end{minipage}\hfill
\end{figure}

\subsection{Cenas fornecidas pela docência da UC}

A docência da UC forneceu, juntamente com o enunciado do trabalho, algumas cenas a serem testadas no
trabalho. A \texttt{engine} renderizou-as como esperado:

\begin{figure}[H]
    \begin{adjustwidth}{-2cm}{-2cm}
        \centering
        \includegraphics[width=0.26\textwidth]{res/phase2/results/Test1.png}
        \includegraphics[width=0.26\textwidth]{res/phase2/results/Test2.png}
        \includegraphics[width=0.26\textwidth]{res/phase2/results/Test3.png}
        \includegraphics[width=0.26\textwidth]{res/phase2/results/Test4.png}
        \caption{Renderização das cenas de teste fornecidas pela docência da UC.}
    \end{adjustwidth}
\end{figure}

\subsection{Sistema Solar}

A hierarquia implementada produziu com sucesso as órbitas e relações de escala do sistema solar.

\begin{figure}[H]
    \centering
    \includegraphics[width=0.5\textwidth]{res/phase2/results/SolarSystem.png}
    \caption{Renderização do sistema solar.}
\end{figure}

\subsection{Frustum Culling}

O \textit{frustum culling} utiliza as esferas vermelhas para verificar se um objeto
se encontra dentro do campo de visão da câmara. Se a esfera estiver dentro da área visível, o
modelo correspondente é renderizado. Os objetos que se encontrem fora do \textit{frustum} têm as
suas esferas ignoradas, o que evita um processamento desnecessário.

\begin{figure}[H]
    \begin{adjustwidth}{-2cm}{-2cm}
        \centering
        \includegraphics[width=0.4\textwidth]{res/phase2/results/FrustumCulling.png}
        \includegraphics[width=0.4\textwidth]{res/phase2/results/FrustumCullingpt2.png}
        \caption{Frustum Culling}
    \end{adjustwidth}
\end{figure}

\section{Conclusão e Trabalho Futuro}

\textbf{\color{red} TODO - conclusão}

\begingroup
\section{Bibliografia}
\renewcommand{\section}[2]{}

\begin{thebibliography}{9}
    \bibitem{lighthouse3d-frustum-planes}
        ``Geometric Approach -- Extracting the Planes.''. Lighthouse3d.com. Accessed:
        Mar. 29, 2025. [Online.] Available:
        \url{https://www.lighthouse3d.com/tutorials/view-frustum-culling/geometric-approach-extracting-the-planes/}
    \bibitem{lighthouse3d-frustum-distances}
        ``View Frustum’s Shape.''. Lighthouse3d.com. Accessed:
        Mar. 29, 2025. [Online.] Available:
        \url{https://www.lighthouse3d.com/tutorials/view-frustum-culling/view-frustums-shape/}
    \bibitem{lighthouse3d-plane}
        ``Plane.''. Lighthouse3d.com. Accessed:
        Mar. 29, 2025. [Online.] Available:
        \url{https://www.lighthouse3d.com/tutorials/maths/plane/}
    \bibitem{lighthouse3d-sphere}
        ``Geometric Approach -- Testing Points and Spheres.''. Lighthouse3d.com. Accessed:
        Mar. 29, 2025. [Online.] Available:
        \url{https://www.lighthouse3d.com/tutorials/view-frustum-culling/geometric-approach-testing-points-and-spheres/}
\end{thebibliography}
\endgroup

\end{document}
